\documentclass[12pt,letterpaper]{article}
\usepackage[top=2cm, bottom=4.5cm, left=2.5cm, right=2.5cm]{geometry}
\usepackage[utf8]{inputenc}
\usepackage[T2A]{fontenc} 
\usepackage[russian]{babel}

%\usepackage{lastpage}
\usepackage{fancyhdr}
\usepackage{mathrsfs}
\usepackage{xcolor}
\usepackage{graphicx}
\usepackage[urlcolor = blue]{hyperref}
\usepackage{amsmath,amsthm}
\usepackage[ruled]{algorithm2e}


%\usepackage{fontspec}
%\usepackage{unicode-math}
\usepackage{libertinus}
\usepackage[libertine]{newtxmath}
\usepackage{dsfont}
\usepackage{microtype}
\usepackage{sidenotes}
%\usepackage{enumerate}
\usepackage[shortlabels, inline]{enumitem}
\usepackage{nicefrac}
\usepackage[small]{titlesec}

\hypersetup{%
  colorlinks=true,
  linkcolor=blue,
  linkbordercolor={0 0 1}
}
 
\setlength{\parindent}{0.0in}
\setlength{\parskip}{0.05in}
\newcommand\course{Теория кодирования и криптография}
\newcommand\hwnumber{1}                  % <-- homework number
\newcommand\NetIDa{}           % <-- NetID of person #1
%\newcommand\NetIDb{netid12038}           % <-- NetID of person #2 (Comment this line out for problem sets)
\pagestyle{fancyplain}
\headheight 35pt
\lhead{\course} %\NetIDa}
%\lhead{\NetIDa\\\NetIDb}                 % <-- Comment this line out for problem sets (make sure you are person #1)
%\chead{\textbf{\Large Листок \hwnumber}}
\rhead{Индивидуальное задание 2}
\lfoot{}
\cfoot{}
\rfoot{\small\thepage}
\headsep 1.5em


\usepackage{listings}
\definecolor{codegreen}{rgb}{0,0,0.6}
\definecolor{codegray}{rgb}{0.5,0.5,0.5}
\definecolor{codepurple}{rgb}{0.58,0,0.82}
\definecolor{backcolour}{rgb}{0.95,0.95,0.92}
\lstdefinestyle{mystyle}{
    backgroundcolor=\color{backcolour},   
    commentstyle=\color{codegreen},
    keywordstyle=\color{magenta},
    numberstyle=\tiny\color{codegray},
    stringstyle=\color{codepurple},
    basicstyle=\ttfamily\footnotesize,
    breakatwhitespace=false,         
    breaklines=true,                 
    captionpos=b,                    
    keepspaces=true,                 
    numbers=left,                    
    numbersep=5pt,                  
    showspaces=false,                
    showstringspaces=false,
    showtabs=false,                  
    tabsize=2,
	texcl=true
}
\lstset{style=mystyle}


\newcommand{\ZZ}{\mathds{Z}}
\newcommand{\NN}{\mathds{N}}
\newcommand{\QQ}{\mathds{Q}}
\newcommand{\FF}{\mathds{F}}
\newtheorem*{remark}{Замечание}

\begin{document}
\textbf{Дедлайн: 12 мая}

\begin{enumerate}[\bfseries 1.]
  \item \emph{(7 баллов)} Пусть $(\alpha_1, \dots, \alpha_n)$ --- набор различных ненулевых элементов конечного поля $\FF_q$ (можно на выбор $q = 1024$ или $q=1031$). Реализуйте
  \begin{enumerate}[\it (a)]
    \item алгоритм кодирования сообщений $(m_0, \dots, m_{k-1}) \in \FF_q$ с помощью кода Рида--Соломона $RS_k(\alpha_1, \dots, \alpha_n)$;
    \item алгоритм генерации и добавления случайной ошибки $\mathbf{e}$ веса $t$ к закодированному сообщению;
  \end{enumerate}  
  На свой выбор реализуйте и протестируйте корректность работы любого из декодеров кодов Рида--Соломона \emph{(например, декодер Велча--Берлекэмпа, декодер Шиозаки--Гао, декодер Берлекэмпа--Мэсси)}.
  
  Рекомендуется использовать уже готовые реализации арифметики в конечных полях и арифметики многочленов (например, SageMath, SymPy, CoCoA, Wolfram Mathematica)

  \item \emph{(5 баллов)} Реализуйте и протестируйте протокол биометрической криптографии \emph{Fuzzy Vault} на основе кодов Рида--Соломона, а именно функции \textsf{LOCK} и \textsf{UNLOCK}.
  
  \item \emph{(5 баллов)} Закодируйте кодом Рида--Маллера $RM(r=2,m=4)$ информационный вектор $\mathbf{m}$, внесите одну ошибку в получившееся кодовое слово, а затем пошагово его декодируйте.
  \begin{remark}
    Пусть ваш месяц и день рождения записаны в виде \emph{MMDD} (например, 0326, 26 марта). Тогда
    информационный вектор $\mathbf{m}$ следует считать равным двоичному представлению этой
    последовательности (без учёта незначащих нулей). Если в получившейся двоичной последовательности меньше $11$ символов, добавьте необходимое количество нулей в начале слова. Пример: $\text{26 марта} \mapsto 0326 \mapsto 00101000110$.
  \end{remark}



  \emph{Бонусные баллы:} 5 баллов первым 10 сдавшим, 3 балла сдавшим до 25 апреля. Также бонусные баллы можно получить за реализацию алгоритмов на компилируемых языках программирования (Rust, C++, C), а также за программную реализацию декодера кодов Рида--Маллера.

  % Пусть Ваш месяц и день рождения записаны в виде MMDD (например, 0326, 26 марта). Тогда
  % информационный вектор 𝑚 следует считать равным двоичному представлению этой
  % последовательности
  % (без
  % учёта
  % незначащих
  % нулей).
  % Если
  % в
  % получившейся
  % двоичной
  % последовательности меньше 11 символов, добавьте необходимое количество нулей в начале слова.
  % Пример. 26 марта = 0326 = 00101000110.



\end{enumerate}

\begin{algorithm}[h]
  \caption{Декодер Велча--Берлекэмпа}
  \KwIn{$(z_1, \dots, z_n)$ --- сообщение, принятое по каналу связи}
  \begin{enumerate}[1.]
    \item Пусть $r = \lfloor \frac{n-k}{2} \rfloor$ --- максимальный вес ошибки, которую можно исправить. Найти многочлены 
    \[ L(x) = \sum_{i=0}^{r} L_i x^i, \quad N(x) = \sum_{i=0}^{r+k-1}N_i x^i,  \]
    которые удовлетворяющие следующей ОСЛУ
    \[
      \begin{cases}
        N(\alpha_1) = z_1 L(\alpha_1) \\
        N(\alpha_2) = z_2 L(\alpha_2) \\
        \dots \\
        N(\alpha_n) = z_n L(\alpha_n) \\
      \end{cases}
    \]
    \item $m(x) = N(x)/L(x)$ --- исходное сообщение
  \end{enumerate}
\end{algorithm}
\begin{algorithm}[h]
  \caption{Декодер Шиозаки--Гао}
  \KwIn{$(z_1, \dots, z_n)$ --- сообщение, принятое по каналу связи}
  \begin{enumerate}[1.]
    \item Построить с помощью интерполяции многочлен $T(x)$, $\deg(T) < n$, такой что
    $T(\alpha_j) = y_j$ для $1 \leq j \leq n$.
    \item С помощью расширенного алгоритма Евклида для  найти многочлены $L(X)$ и $N(x)$, что
    \[ 
      \begin{cases}
        L(X)T(x) \equiv N(X) \; (\operatorname{mod} W(x)), & \text{где} \; W(x) = \prod_{i=1}^{n} (x - \alpha_i) \\
        \deg N(x) \leq r+k-1, \\
        \deg L(x) \to \max
      \end{cases}
    \]
    
    \item $m(x) = N(x)/L(x)$ --- исходное сообщение
  \end{enumerate}
\end{algorithm}

\begin{algorithm}[h]
  \caption{LOCK}
  \KwIn{$A = \{\alpha_1, \dots, \alpha_n \} \subset \FF_q^*$ --- образец биометрии, $m(x)$ --- секретный многочлен небольшой степени}
  $S \gets Dict()$\;
  \ForEach{$\beta \in \FF_q^*$}{
    \uIf{$\beta \in A$}{
      $S[\beta] \gets m(\beta)$\;
    }\uElse{
      $S[\beta] \gets rand()$\;
    }
  }
  \Return{S}\;
\end{algorithm}

\end{document}